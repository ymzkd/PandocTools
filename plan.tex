\subsection{アプリ概要}\label{ux30a2ux30d7ux30eaux6982ux8981}

\begin{itemize}
\item
  \textbf{名称(仮)}:Pandoc GUI Converter
\item
  \textbf{目的}:Windows 上で Markdown ファイルを選択し、pypandoc 経由で PDF(HTML/DOCX も可)に変換するシンプルなデスクトップ GUI アプリケーション。
\item
  \textbf{特徴}:

  \begin{itemize}
  \tightlist
  \item
    Rust製のパッケージ管理ツール \textbf{uv} で依存パッケージ・Pythonバージョン・仮想環境を一元管理し、環境構築を爆速化 ({[}gihyo.jp{]}{[}1{]})。
  \item
    PyQt6 による直感的な UI(ファイル選択、オプション設定、ログ表示)を提供。
  \item
    Pandoc の全オプションを透過的に指定できる \texttt{extra\_args} 編集機能。
  \item
    \texttt{QProcess} を用いた非同期起動で UI をブロックしない。
  \end{itemize}
\end{itemize}

\begin{center}\rule{0.5\linewidth}{0.5pt}\end{center}

\subsection{要件}\label{ux8981ux4ef6}

\subsubsection{機能要件}\label{ux6a5fux80fdux8981ux4ef6}

\begin{enumerate}
\def\labelenumi{\arabic{enumi}.}
\item
  \textbf{入力管理}

  \begin{itemize}
  \tightlist
  \item
    Markdown ファイル/フォルダの選択ダイアログ
  \item
    ドラッグ&ドロップによるファイル追加対応
  \end{itemize}
\item
  \textbf{出力形式選択}

  \begin{itemize}
  \tightlist
  \item
    プルダウンで「pdf」「html」「docx」等を選択
  \end{itemize}
\item
  \textbf{オプション設定}

  \begin{itemize}
  \tightlist
  \item
    チェックボックス・テキスト欄で Pandoc オプション(\texttt{-\/-wrap=preserve}、\texttt{-\/-lua-filter}、\texttt{-\/-pdf-engine=xelatex}、\texttt{-V\ documentclass=…}、\texttt{-\/-from\ markdown+hard\_line\_breaks} 等)を編集
  \end{itemize}
\item
  \textbf{プロファイル管理}

  \begin{itemize}
  \tightlist
  \item
    YAML 形式での保存/読み込み
  \item
    インストール先直下の \texttt{profiles/} フォルダに配置
  \end{itemize}
\item
  \textbf{変換実行}

  \begin{itemize}
  \tightlist
  \item
    \texttt{QProcess} で Pandoc CLI を非同期起動し、標準出力/標準エラーをリアルタイム表示
  \item
    終了時に「出力先を開く」ボタンを有効化
  \end{itemize}
\item
  \textbf{ログ永続化}

  \begin{itemize}
  \tightlist
  \item
    不要(ファイル保存なし)
  \end{itemize}
\end{enumerate}

\subsubsection{非機能要件}\label{ux975eux6a5fux80fdux8981ux4ef6}

\begin{itemize}
\tightlist
\item
  \textbf{プラットフォーム}:Windows のみ対応(将来的にクロスプラットフォーム化検討)
\item
  \textbf{仮想環境管理}:uv を利用し、一貫して依存・Python バージョン・仮想環境を管理 ({[}gihyo.jp{]}{[}1{]})。
\item
  \textbf{言語}:日本語 UI 固定
\item
  \textbf{軽量性}:依存は最小限(PyQt6, pypandoc, pyyaml, uv)
\item
  \textbf{拡張性}:後日自動更新機能や多言語対応を追加可能な設計
\end{itemize}

\begin{center}\rule{0.5\linewidth}{0.5pt}\end{center}

\subsection{実装に必要な情報}\label{ux5b9fux88c5ux306bux5fc5ux8981ux306aux60c5ux5831}

\subsubsection{1. 環境セットアップ(uv 活用)}\label{ux74b0ux5883ux30bbux30c3ux30c8ux30a2ux30c3ux30d7uv-ux6d3bux7528}

\begin{enumerate}
\def\labelenumi{\arabic{enumi}.}
\item
  \textbf{uv のインストール}

\begin{Shaded}
\begin{Highlighting}[]
\CommentTok{\# システムに pip があれば}
\NormalTok{pip install uv}
\end{Highlighting}
\end{Shaded}

  Rust 製で極めて高速な pip 代替およびプロジェクト管理ツール ({[}gihyo.jp{]}{[}1{]})。
\item
  \textbf{プロジェクト初期化}

\begin{Shaded}
\begin{Highlighting}[]
\NormalTok{uv init pandoc{-}gui}
\FunctionTok{cd}\NormalTok{ pandoc{-}gui}
\end{Highlighting}
\end{Shaded}

  \texttt{pyproject.toml} 等のひな形が生成される ({[}speakerdeck.com{]}{[}2{]})。
\item
  \textbf{仮想環境の作成}

\begin{Shaded}
\begin{Highlighting}[]
\NormalTok{uv venv}
\end{Highlighting}
\end{Shaded}

  デフォルトで \texttt{.venv} フォルダを 0.02 秒程度で作成 ({[}gihyo.jp{]}{[}1{]})。
\item
  \textbf{仮想環境のアクティベート}

\begin{Shaded}
\begin{Highlighting}[]
\OperatorTok{.}\NormalTok{\textbackslash{}}\OperatorTok{.}\FunctionTok{venv}\NormalTok{\textbackslash{}Scripts\textbackslash{}Activate}\OperatorTok{.}\FunctionTok{ps1}
\end{Highlighting}
\end{Shaded}

  (PowerShell)Windows 環境でも通常の venv と同様に有効化可能 ({[}gihyo.jp{]}{[}1{]})。
\item
  \textbf{依存パッケージのインストール}

\begin{Shaded}
\begin{Highlighting}[]
\NormalTok{uv pip install PyQt6 pypandoc pyyaml}
\end{Highlighting}
\end{Shaded}

  \texttt{uv\ pip} コマンドは従来の \texttt{pip\ install} より高速・確実にパッケージを解決・インストール ({[}gihyo.jp{]}{[}1{]})。
\end{enumerate}

\subsubsection{2. ディレクトリ構成例}\label{ux30c7ux30a3ux30ecux30afux30c8ux30eaux69cbux6210ux4f8b}

\begin{verbatim}
pandoc-gui/                      ← uv init で生成されたルート
├─ .venv/                        ← uv venv による仮想環境
├─ profiles/                     ← YAML プロファイル保存 (例: default.yml)
├─ src/
│   ├─ main.py                   ← アプリ起動/UI初期化
│   ├─ ui_main.py                ← Qt Designer 生成 or 手書き UI 定義
│   ├─ pandoc_process.py         ← QProcess を使った非同期 Pandoc 実行
│   ├─ config.py                 ← YAML プロファイル読み書きロジック
│   └─ resources/                ← アイコン等リソース
├─ pyproject.toml                ← uv init 生成
├─ README.md
└─ requirements.txt              ← uv pip freeze 出力(任意)
\end{verbatim}

\subsubsection{3. 主なコード例}\label{ux4e3bux306aux30b3ux30fcux30c9ux4f8b}

\paragraph{\texorpdfstring{\texttt{config.py}(YAML プロファイル管理)}{config.py(YAML プロファイル管理)}}\label{config.pyyaml-ux30d7ux30edux30d5ux30a1ux30a4ux30ebux7ba1ux7406}

\begin{Shaded}
\begin{Highlighting}[]
\ImportTok{import}\NormalTok{ yaml}
\ImportTok{from}\NormalTok{ pathlib }\ImportTok{import}\NormalTok{ Path}
\ImportTok{import}\NormalTok{ sys}

\NormalTok{BASE\_DIR }\OperatorTok{=}\NormalTok{ Path(sys.argv[}\DecValTok{0}\NormalTok{]).resolve().parent}
\NormalTok{PROFILE\_DIR }\OperatorTok{=}\NormalTok{ BASE\_DIR }\OperatorTok{/} \StringTok{\textquotesingle{}profiles\textquotesingle{}}
\NormalTok{PROFILE\_DIR.mkdir(exist\_ok}\OperatorTok{=}\VariableTok{True}\NormalTok{)}

\KeywordTok{def}\NormalTok{ load\_profile(name: }\BuiltInTok{str}\NormalTok{) }\OperatorTok{{-}\textgreater{}} \BuiltInTok{dict}\NormalTok{:}
    \ControlFlowTok{with} \BuiltInTok{open}\NormalTok{(PROFILE\_DIR }\OperatorTok{/} \SpecialStringTok{f\textquotesingle{}}\SpecialCharTok{\{}\NormalTok{name}\SpecialCharTok{\}}\SpecialStringTok{.yml\textquotesingle{}}\NormalTok{, }\StringTok{\textquotesingle{}r\textquotesingle{}}\NormalTok{, encoding}\OperatorTok{=}\StringTok{\textquotesingle{}utf{-}8\textquotesingle{}}\NormalTok{) }\ImportTok{as}\NormalTok{ f:}
        \ControlFlowTok{return}\NormalTok{ yaml.safe\_load(f)}

\KeywordTok{def}\NormalTok{ save\_profile(name: }\BuiltInTok{str}\NormalTok{, data: }\BuiltInTok{dict}\NormalTok{):}
    \ControlFlowTok{with} \BuiltInTok{open}\NormalTok{(PROFILE\_DIR }\OperatorTok{/} \SpecialStringTok{f\textquotesingle{}}\SpecialCharTok{\{}\NormalTok{name}\SpecialCharTok{\}}\SpecialStringTok{.yml\textquotesingle{}}\NormalTok{, }\StringTok{\textquotesingle{}w\textquotesingle{}}\NormalTok{, encoding}\OperatorTok{=}\StringTok{\textquotesingle{}utf{-}8\textquotesingle{}}\NormalTok{) }\ImportTok{as}\NormalTok{ f:}
\NormalTok{        yaml.safe\_dump(data, f, allow\_unicode}\OperatorTok{=}\VariableTok{True}\NormalTok{)}
\end{Highlighting}
\end{Shaded}

\paragraph{\texorpdfstring{\texttt{pandoc\_process.py}(QProcess で非同期 Pandoc 実行)}{pandoc\_process.py(QProcess で非同期 Pandoc 実行)}}\label{pandoc_process.pyqprocess-ux3067ux975eux540cux671f-pandoc-ux5b9fux884c}

\begin{Shaded}
\begin{Highlighting}[]
\ImportTok{from}\NormalTok{ PyQt6.QtCore }\ImportTok{import}\NormalTok{ QObject, QProcess, pyqtSignal}

\KeywordTok{class}\NormalTok{ PandocWorker(QObject):}
\NormalTok{    stdout\_received }\OperatorTok{=}\NormalTok{ pyqtSignal(}\BuiltInTok{str}\NormalTok{)}
\NormalTok{    stderr\_received }\OperatorTok{=}\NormalTok{ pyqtSignal(}\BuiltInTok{str}\NormalTok{)}
\NormalTok{    finished }\OperatorTok{=}\NormalTok{ pyqtSignal(}\BuiltInTok{int}\NormalTok{)}

    \KeywordTok{def} \FunctionTok{\_\_init\_\_}\NormalTok{(}\VariableTok{self}\NormalTok{, parent}\OperatorTok{=}\VariableTok{None}\NormalTok{):}
        \BuiltInTok{super}\NormalTok{().}\FunctionTok{\_\_init\_\_}\NormalTok{(parent)}
        \VariableTok{self}\NormalTok{.proc }\OperatorTok{=}\NormalTok{ QProcess(}\VariableTok{self}\NormalTok{)}
        \VariableTok{self}\NormalTok{.proc.readyReadStandardOutput.}\ExtensionTok{connect}\NormalTok{(}\VariableTok{self}\NormalTok{.\_on\_stdout)}
        \VariableTok{self}\NormalTok{.proc.readyReadStandardError.}\ExtensionTok{connect}\NormalTok{(}\VariableTok{self}\NormalTok{.\_on\_stderr)}
        \VariableTok{self}\NormalTok{.proc.finished.}\ExtensionTok{connect}\NormalTok{(}\VariableTok{self}\NormalTok{.\_on\_finished)}

    \KeywordTok{def}\NormalTok{ run(}\VariableTok{self}\NormalTok{, input\_md: }\BuiltInTok{str}\NormalTok{, output\_file: }\BuiltInTok{str}\NormalTok{, extra\_args: }\BuiltInTok{list}\NormalTok{[}\BuiltInTok{str}\NormalTok{]):}
\NormalTok{        cmd }\OperatorTok{=}\NormalTok{ [}\StringTok{\textquotesingle{}pandoc\textquotesingle{}}\NormalTok{, input\_md, }\StringTok{\textquotesingle{}{-}o\textquotesingle{}}\NormalTok{, output\_file] }\OperatorTok{+}\NormalTok{ extra\_args}
        \VariableTok{self}\NormalTok{.proc.start(cmd[}\DecValTok{0}\NormalTok{], cmd[}\DecValTok{1}\NormalTok{:])}

    \KeywordTok{def}\NormalTok{ \_on\_stdout(}\VariableTok{self}\NormalTok{):}
        \VariableTok{self}\NormalTok{.stdout\_received.emit(}\BuiltInTok{bytes}\NormalTok{(}\VariableTok{self}\NormalTok{.proc.readAllStandardOutput()).decode())}

    \KeywordTok{def}\NormalTok{ \_on\_stderr(}\VariableTok{self}\NormalTok{):}
        \VariableTok{self}\NormalTok{.stderr\_received.emit(}\BuiltInTok{bytes}\NormalTok{(}\VariableTok{self}\NormalTok{.proc.readAllStandardError()).decode())}

    \KeywordTok{def}\NormalTok{ \_on\_finished(}\VariableTok{self}\NormalTok{, exit\_code: }\BuiltInTok{int}\NormalTok{, \_status):}
        \VariableTok{self}\NormalTok{.finished.emit(exit\_code)}
\end{Highlighting}
\end{Shaded}

\paragraph{\texorpdfstring{\texttt{main.py}(UI と連携)}{main.py(UI と連携)}}\label{main.pyui-ux3068ux9023ux643a}

\begin{Shaded}
\begin{Highlighting}[]
\ImportTok{import}\NormalTok{ sys}
\ImportTok{from}\NormalTok{ PyQt6.QtWidgets }\ImportTok{import}\NormalTok{ QApplication, QMainWindow, QFileDialog}
\ImportTok{from}\NormalTok{ ui\_main }\ImportTok{import}\NormalTok{ Ui\_MainWindow}
\ImportTok{from}\NormalTok{ pandoc\_process }\ImportTok{import}\NormalTok{ PandocWorker}
\ImportTok{from}\NormalTok{ config }\ImportTok{import}\NormalTok{ load\_profile, save\_profile}

\KeywordTok{class}\NormalTok{ MainWindow(QMainWindow):}
    \KeywordTok{def} \FunctionTok{\_\_init\_\_}\NormalTok{(}\VariableTok{self}\NormalTok{):}
        \BuiltInTok{super}\NormalTok{().}\FunctionTok{\_\_init\_\_}\NormalTok{()}
        \VariableTok{self}\NormalTok{.ui }\OperatorTok{=}\NormalTok{ Ui\_MainWindow()}
        \VariableTok{self}\NormalTok{.ui.setupUi(}\VariableTok{self}\NormalTok{)}

        \VariableTok{self}\NormalTok{.worker }\OperatorTok{=}\NormalTok{ PandocWorker()}
        \VariableTok{self}\NormalTok{.worker.stdout\_received.}\ExtensionTok{connect}\NormalTok{(}\VariableTok{self}\NormalTok{.ui.appendLog)}
        \VariableTok{self}\NormalTok{.worker.stderr\_received.}\ExtensionTok{connect}\NormalTok{(}\VariableTok{self}\NormalTok{.ui.appendLog)}
        \VariableTok{self}\NormalTok{.worker.finished.}\ExtensionTok{connect}\NormalTok{(}\VariableTok{self}\NormalTok{.onFinished)}

        \VariableTok{self}\NormalTok{.ui.btnSelectInput.clicked.}\ExtensionTok{connect}\NormalTok{(}\VariableTok{self}\NormalTok{.selectInput)}
        \VariableTok{self}\NormalTok{.ui.btnRun.clicked.}\ExtensionTok{connect}\NormalTok{(}\VariableTok{self}\NormalTok{.startConversion)}
        \VariableTok{self}\NormalTok{.ui.btnSaveProfile.clicked.}\ExtensionTok{connect}\NormalTok{(}\VariableTok{self}\NormalTok{.saveCurrentProfile)}
        \VariableTok{self}\NormalTok{.ui.btnLoadProfile.clicked.}\ExtensionTok{connect}\NormalTok{(}\VariableTok{self}\NormalTok{.loadProfile)}

    \KeywordTok{def}\NormalTok{ selectInput(}\VariableTok{self}\NormalTok{):}
\NormalTok{        path, \_ }\OperatorTok{=}\NormalTok{ QFileDialog.getOpenFileName(}\VariableTok{self}\NormalTok{, }\StringTok{"Markdown を選択"}\NormalTok{, }\StringTok{""}\NormalTok{, }\StringTok{"Markdown (*.md)"}\NormalTok{)}
        \ControlFlowTok{if}\NormalTok{ path:}
            \VariableTok{self}\NormalTok{.ui.inputPath.setText(path)}

    \KeywordTok{def}\NormalTok{ startConversion(}\VariableTok{self}\NormalTok{):}
\NormalTok{        in\_md }\OperatorTok{=} \VariableTok{self}\NormalTok{.ui.inputPath.text()}
\NormalTok{        out\_file }\OperatorTok{=} \VariableTok{self}\NormalTok{.ui.outputPath.text()}
\NormalTok{        args }\OperatorTok{=} \VariableTok{self}\NormalTok{.ui.collectExtraArgs()}
        \VariableTok{self}\NormalTok{.worker.run(in\_md, out\_file, args)}

    \KeywordTok{def}\NormalTok{ onFinished(}\VariableTok{self}\NormalTok{, code):}
        \ControlFlowTok{if}\NormalTok{ code }\OperatorTok{==} \DecValTok{0}\NormalTok{:}
            \VariableTok{self}\NormalTok{.ui.enableOpenOutput(}\VariableTok{True}\NormalTok{)}

    \KeywordTok{def}\NormalTok{ saveCurrentProfile(}\VariableTok{self}\NormalTok{):}
\NormalTok{        name }\OperatorTok{=} \VariableTok{self}\NormalTok{.ui.profileName.text()}
\NormalTok{        data }\OperatorTok{=}\NormalTok{ \{}
            \StringTok{"output\_format"}\NormalTok{: }\VariableTok{self}\NormalTok{.ui.outputFormat.currentText(),}
            \StringTok{"extra\_args"}\NormalTok{: }\VariableTok{self}\NormalTok{.ui.collectExtraArgs()}
\NormalTok{        \}}
\NormalTok{        save\_profile(name, data)}

    \KeywordTok{def}\NormalTok{ loadProfile(}\VariableTok{self}\NormalTok{):}
\NormalTok{        name }\OperatorTok{=} \VariableTok{self}\NormalTok{.ui.profileSelect.currentText()}
\NormalTok{        data }\OperatorTok{=}\NormalTok{ load\_profile(name)}
        \CommentTok{\# UI に data を反映する実装を追加}

\ControlFlowTok{if} \VariableTok{\_\_name\_\_} \OperatorTok{==} \StringTok{\textquotesingle{}\_\_main\_\_\textquotesingle{}}\NormalTok{:}
\NormalTok{    app }\OperatorTok{=}\NormalTok{ QApplication(sys.argv)}
\NormalTok{    win }\OperatorTok{=}\NormalTok{ MainWindow()}
\NormalTok{    win.show()}
\NormalTok{    sys.exit(app.}\BuiltInTok{exec}\NormalTok{())}
\end{Highlighting}
\end{Shaded}

\subsubsection{4. プロファイル YAML 例}\label{ux30d7ux30edux30d5ux30a1ux30a4ux30eb-yaml-ux4f8b}

\begin{Shaded}
\begin{Highlighting}[]
\FunctionTok{output\_format}\KeywordTok{:}\AttributeTok{ pdf}
\FunctionTok{extra\_args}\KeywordTok{:}
\AttributeTok{  }\KeywordTok{{-}}\AttributeTok{ {-}{-}wrap=preserve}
\AttributeTok{  }\KeywordTok{{-}}\AttributeTok{ {-}{-}lua{-}filter=strip\_math\_blocks.lua}
\AttributeTok{  }\KeywordTok{{-}}\AttributeTok{ {-}{-}pdf{-}engine=xelatex}
\AttributeTok{  }\KeywordTok{{-}}\AttributeTok{ {-}V}
\AttributeTok{  }\KeywordTok{{-}}\AttributeTok{ documentclass=bxjsarticle}
\AttributeTok{  }\KeywordTok{{-}}\AttributeTok{ {-}V}
\AttributeTok{  }\KeywordTok{{-}}\AttributeTok{ classoption=pandoc}
\AttributeTok{  }\KeywordTok{{-}}\AttributeTok{ {-}{-}from}
\AttributeTok{  }\KeywordTok{{-}}\AttributeTok{ markdown+hard\_line\_breaks}
\end{Highlighting}
\end{Shaded}

\subsubsection{5. ビルド \& 配布(Windows 向け)}\label{ux30d3ux30ebux30c9-ux914dux5e03windows-ux5411ux3051}

\begin{Shaded}
\begin{Highlighting}[]
\NormalTok{uv pip install pyinstaller}
\NormalTok{pyinstaller \textasciigrave{}}
  \OperatorTok{{-}{-}}\NormalTok{name pandoc{-}gui \textasciigrave{}}
  \OperatorTok{{-}{-}}\NormalTok{onefile \textasciigrave{}}
  \OperatorTok{{-}{-}}\NormalTok{add{-}data }\StringTok{"profiles;profiles"}\NormalTok{ \textasciigrave{}}
\NormalTok{  src}\OperatorTok{/}\NormalTok{main}\OperatorTok{.}\FunctionTok{py}
\end{Highlighting}
\end{Shaded}

\begin{itemize}
\tightlist
\item
  生成された \texttt{dist/pandoc-gui.exe} を配布先に配置するだけで動作 ({[}js2iiu.com{]}{[}3{]})。
\end{itemize}
